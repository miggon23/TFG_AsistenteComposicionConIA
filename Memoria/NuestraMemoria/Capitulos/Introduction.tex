\chapter*{Introduction}
\label{cap:introduction}
\addcontentsline{toc}{chapter}{Introduction}

Introduction to the subject area. This chapter contains the translation of Chapter \ref{cap:introduccion}.

\section{Motivación}
Our main purpose for this project is to make music composition, focused primarily on video games, to users with low or non existent musical knowledge.

In order to reach this purpose, we want to rely on the field of Artificial Intelligence, exploring techniques and methodologies to be able to generate something that is interesting.

\section{Objectives}
With our purpose defined, the next steps is to define a series of objectives:
\begin{enumerate}
    \item Generate melodies through the use of different tools, including \textit{Machine Learning} or stochastic algorithms.
    \item Harmonize the melody previously generated in a way that results pleasant to the human ear, either through the use of Artificial Intelligence or own heuristics.
    \item Sound the generated music notes making a property selection of instruments, choosing between pre-defined themes. 
    \item Create an application for \textit{Windows} that allows the user to apply this techniques in a friendly way.
\end{enumerate}

\section{Plan de trabajo}
El tener objetivos claramente separados en cuanto a funcionalidad, nos permite separar el trabajo de manera más eficiente en módulos independientes, como una cadena de producción separada en etapas que se une en la app final. 

\begin{enumerate}
    \item Investigación y estudio del campo de la Inteligencia Artificial aplicada a la generación musical. Durante esta fase, revisaremos la literatura existente, investigaremos los enfoques y algoritmos más relevantes y nos familiarizaremos con las herramientas y recursos disponibles.
    \item Desarrollo del algoritmo de generación de melodías. Utilizaremos técnicas de aprendizaje automático y modelado generativo para crear un sistema capaz de componer melodías originales y variadas.
    \item Implementación de la armonización de las melodías generadas. En esta etapa, diseñaremos y desarrollaremos heurísticas y reglas musicales para armonizar las melodías de manera coherente y agradable al oído.
    \item Selección y configuración de instrumentos. Investigaremos y seleccionaremos una variedad de instrumentos musicales virtuales que se ajusten a las temáticas predefinidas de los videojuegos. Configuraremos estos instrumentos para que reproduzcan las melodías generadas de manera convincente al oído humano.
    \item Desarrollo de la aplicación de escritorio. Utilizaremos las tecnologías adecuadas para desarrollar una interfaz de usuario intuitiva y amigable que permita a los usuarios generar, armonizar y sonorizar sus propias composiciones musicales para videojuegos.
\end{enumerate}







