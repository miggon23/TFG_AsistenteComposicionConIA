\chapter{Interfaz de Usuario con TKInter}
\label{cap:GUIconTKInter}
Hemos elegido TKINter como frontend para nuestra aplicación. El frontend es la parte visible de una aplicación, con la que puede interactuar el usuario.

\section{TKInter}
\label{sec:TK}
\href{https://docs.python.org/es/3/library/tkinter.html}{TKInter} es la biblioteca por defecto que trae Python para la creación de GUI, interfaaz gráfica de usuario por sus siglas en inglés.

Al venir integrado con python, es una herramienta de fácil inclusión en los proyectos que usen python. TKInter brinda una interfaz fácil de usar, pero muy sencilla en cuanto a posibilidades. Hemos priorizado la funcionalidad frente a la estética, política que casa muy bien con TKInter.

\subsection{Widgets de TKInter}
Los \textit{widgets} de TKInter son elementos de UI a los que se les puede vincular acciones de código propio. Estos \textit{widgets} van asociados a una ventana para poder renderizarse en pantalla y así recibir acciones del usuario o mostrar información. Veremos en las subsecciones siguientes las funciones de la aplicación, los widgets asociados y el funcionamiento de nuestra aplicación por debajo.

\subsection{Sistema de pestañas con Tk.Notebook}
\textit{Tk.Notebook} es un \textit{widget} que permite tener varias pestañas en la aplicación. Esto se hace creando una nueva ventana y pasándosela al \textit{Notebook} para que las organice. Esto nos deja la posibilidad de organizar y distribuir las funcionalidades de la aplicación en distintas pestañas como son: \generationTabName{}, \tematicTabName{}, \advancedTabName{}, \configTabName{}.

% Se puede cambiar el título del capítulo por La App y dejar la GUI como sección
\section{La app}
\label{sec:TK:app}

\subsection{Generación Musical}
Pestaña de la aplicación que permite generar el MIDI. Esto se realiza vinculando los \textit{widgets} de botón de TKInter con los apartados de generación y armonía.

** referencias a generación de Rodrigo y armonización de VM **

\subsection{Temáticas y sonorización}
\label{subsec:TK:TematicasYSonorizacion}
La pestaña \tematicTabName{} anexiona las principales funcionalidades de la aplicación. El MIDI generado en la pestaña \generationTabName{} se sonoriza de acuerdo a la configuración de la pestaña \tematicTabName{}. 

**referencias a Sonorización de Javi**

\subsubsection{Semilla}
La semilla es la fuente de generador aleatoria que permite obtener los mismos valores cuando se obtiene un nuevo \textit{random()} de un generador.

Nos interesa poder generar distintas semillas, que den variedad a los resultados generados por la aplicación. Conociendo que semilla se ha generado, podemos guardarlas (ver \nameref{subsub:TK:serializacionYPresets}).

** referncia a arreglos en REAPER de Javi **

\subsubsection{Serialización y presets}
\label{subsub:TK:serializacionYPresets}
La aplicación cuenta con la posibilidad de guardar su estado en momentos específicos. Esto lo hacemos leyendo los atributos de una clase llamada AppModeState y serializándolos a un objeto Json. Una vez tenemos este objeto Json, lo persistimos a un fichero .json con un nombre que nos de el usuario. Estos ficheros que guardan la configuración del estado de la aplicación reciben el nombre de \textit{presets}. 

Estos presets pueden ser recuperados para establecer el estado de la aplicación a la información contenida en el json. Esta característica permite al usuario guardar la configuración de temática, checkboxes (modificadores) y semillas que quiera recuperar possteriormente.

\subsection{Modo avanzado}
La pestaña \advancedTabName{} de la aplicación dota al usuario de un control más exhaustivo de la aplicación. Esta pestaña está apartada del público general debido a que ofrece una visión más compleja de la aplicación, estando dirigida a usuarios de la aplicación avanzados o personas con conocimientos musicales avanzados.

\subsection{Configuración y guardado}
Debido a que dependemos de software de terceros como es REAPER y no podemos brindarlo con nuestra aplicación, necesitamos que el usuario nos brinde información acerca de su sistema, como en qué ruta se encuentra el ejecutable de REAPER para poder iniciarlo y hacer uso de nuestra aplicación
