\chapter{Manual de Usuario}
\label{cap:descripcionTrabajo}

En este capítulo recogeremos los pasos que debe seguir un usuario para poder utilizar la aplicación. Además de las \refname{sec:dependeciasApp}, se explican los dos grandes pasos que hay que tomar para generar un tema con nuestra aplicación. 
Estos dos grandes pasos son la \nameref{sec:dependeciasApp} y la \nameref{sec:instrumentalizacion} del contenido generado en el primer paso. 

\section{Dependencias de la aplicación}
\label{sec:dependeciasApp}
\subsection{Dependencias de Instalación}
	Para ver todas las dependencias, consultar el Apéndice \ref{Appendix:Key1}

	Para hacer uso de la aplicación, se requiere instalar herramientas de terceros que permiten el uso de nuestro asistente de composición musical.
	Estas dependencias se componen desde lenguajes de programación tales como Python y JavaScript hasta los instrumentos que permiten poder escuchar los temas musicales generados.

\subsection{Interfaz gráfica con TKInter}
	TKInter es la interfaz por defecto que \PythonLink{} ofrece a los desarrolladores para montar una 
    GUI (Interfaz Gráfica de Usuario por sus siglas en ingles).
	Esta herramienta viene integrada junto con Python, por lo que no requiere instalar ningún entorno ni herramienta adicional para hacer uso de ella.



\subsection{REAPER}
\label{subsec:reaper}
	\href{https://www.reaper.fm/}{REAPER} es el DAW de producción musical que usamos en nuestra aplicación
	Una DAW, \textit{Digital Audio Workstation}, es una estación de trabajo de audio digital. Las DAWs permiten grabar y editar audio, así como interpretar música simbólica (principalmente MIDI) para generar sonido.

	Este entorno de edición y producción musical es el que nos permite sonorizar el audio que generamos en nuestra aplicación. Es decir, nos permite agregar los instrumentos y efectos de audio que hacen sonar el MIDI generado o proporcionado por el usuario.
    
    Para el uso de la herramienta debemos instalar Reaper 7, desde su \href{https://www.reaper.fm/download.php}{web oficial}. El usuario debe tener en cuenta que Reaper no es una DAW gratuita, pero permite un periodo de prueba con todas sus funcionalidades y donde podrá usar la herramienta sin ningún tipo de problemas, tan sólo hay que esperar unos segundos y cerrar la pestaña que aparece explicando que estamos usando el periodo de prueba.

    Una vez tengamos Reaper instalado, debemos selecionar el apartado Opciones, Preferencias (o atajo de teclado Ctrl+P) y buscar el apartado ReaScript dentro del apartado Plug-ins. Una vez en esta pestaña, marcaremos la casilla Enable Python for use with ReaScript y le proporcionaremos la ruta a nuestro Python 3.9.0 (Apéndice \ref{Appendix:Key1}), así como el nombre de la dll de Python (python39.dll en el caso de usar la versión que recomendamos) Pulsamos apply y debería de indicarnos que python está instalado correctamente, reiniciamos Reaper y estamos listos para usar la herramienta.

\section{Generación de audio}
\label{sec:generacionAudio}
	Generación de audio
\section{Selección de Temática y Sonorización}
\label{sec:SeleccionTematica}

\section{Modo avanzado}
\label{sec:app:advancedMode}
El objetivo principal de la aplicación es ofrecer a los usuarios una forma sencilla de generar temas musicales. No obstante, está incluida la posibilidad de relizar acciones avanzadas para los usuarios más avezados de la aplicación, así como los que cuentan con conicimiento musical previo.

La posibilidad de editar en REAPER los temas musicales generados por nuestra aplicación sigue existiendo, tal y como está planteado en la sección \nameref{sec:SeleccionTematica}, al pulsar el botón de \textit{Play}. Esto iniciará el ejecutable de REAPER, por lo que en cualquier momento el usuario puede pasar a trabajar directamente en la DAW en lugar de centrarse en nuestra aplicación, permitiendo trabajar con normalidad en REAPER, a partir de los MIDIs e instrumentos generados por \appName{}

En los subapartados siguientes veremos las acciones posibles dentro de la pestaña \advancedTabName{} de la aplicación.

\subsection{Alternar Generador}
\label{subsec:alternarGenerador}
Esta opción permite cambiar el modo de generación de la melodía en la pestaña \generationTabName{}. Los generadores disponibles son \nameref{sec:markov-chain}, \nameref{sec:RNR} y \nameref{sec:magenta}. Ver el capítulo \nameref{cap:generacionMusical} para más información acerca de los modos de generacion musical.

\subsection{Mezclar temáticas}

\subsection{Mezclar melodías}
