\chapter{Manual de Usuario}
\label{cap:descripcionTrabajo}

En este capítulo recogeremos los pasos que debe seguir un usuario para poder utilizar la aplicación. Además de las \refname{sec:dependeciasApp}, se explican los dos grandes pasos que hay que tomar para generar un tema con nuestra aplicación. 

Estos dos grandes pasos son la \nameref{sec:dependeciasApp} y la \nameref{sec:instrumentalizacion} del contenido generado en el primer paso. 

\section{Dependencias de la aplicación}
\label{sec:dependeciasApp}
\subsection{Dependencias de Instalación}
	Para ver todas las dependencias, consultar el Apéndice \ref{Appendix:Key1}

	Para hacer uso de la aplicación, se requiere instalar herramientas de terceros que permiten el uso de nuestro asistente de composición musical.
	Estas dependencias se componen desde lenguajes de programación tales como Python y JavaScript hasta los instrumentos que permiten poder escuchar los temas musicales generados.

\subsection{Interfaz gráfica con TKInter}
	TKInter es la interfaz por defecto que \PythonLink{} ofrece a los desarrolladores para montar una GUI (Interfaz Gráfica de Usuario por sus siglas en ingles).
	Esta herramienta viene integrada junto con Python, por lo que no requiere instalar ningún entorno ni herramienta adicional para hacer uso de ella.



\subsection{REAPER}
\label{subsec:reaper}
	\href{https://www.reaper.fm/}{REAPER} es el DAW de producción musical que usamos en nuestra aplicación
	Un DAW, \textit{Digital Audio Workstation}, es una estación trabajo de audio digital.

	Este entorno de edición y producción musical es el que nos permite sonorizar el audio que generamos en nuestra aplicación. Es decir, nos permite agregar los instrumentos y efectos de audio que hacen sonar el MIDI almacenado. 

\section{Generación de audio}
\label{sec:generacionAudio}
	Generación de audio
\section{Instrumentalizacion}
\label{sec:instrumentalizacion}
	Instrumentalizacion
	