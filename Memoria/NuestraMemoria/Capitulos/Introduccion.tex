\chapter{Introducción}
\label{cap:introduccion}

\chapterquote{Frase célebre dicha por alguien inteligente}{Autor}


\section{Motivación}
Introducción al tema del TFG.


\section{Objetivos}
 Nuestro objetivo en este proyecto es acercar la generación de temas musicales enfocados a videojuegos a usuarios con bajos o nulos conocimientos en composición musical a través del campo de la Inteligencia Artificial.
 De tal modo, el primer objetivo es conseguir generar melodías a través del estudio del campo de la Inteligencia Artificial.
 El siguiente objetivo es conseguir armonizar la melodía generada de forma que resulta agradable al oído humano. En lugar de usar Inteligencia Artificial, en este punto optamos por utilizar heurísticas propias.
 Nuestro tercer objetivo es sonorizar las notas musicales generadas haciendo una selección apropiada de instrumentos en base a unas temásticas predefinidas.
 Como objetivo final, nos proponemos crear una aplicación de escritorio que ofrezca las posibilidades mencionadas anteriormente de una forma amigable al usuario.

\section{Plan de trabajo}
El tener objetivos claramente separados en cuanto a funcionalidad, nos permite separar el trabajo de manera más eficiente en módulos independientes, como una cadena de producción separada en etapas que se une en la app final. 

En primer lugar: Investigación y estudio del campo de la Inteligencia Artificial aplicada a la generación musical. Durante esta fase, revisaremos la literatura existente, investigaremos los enfoques y algoritmos más relevantes y nos familiarizaremos con las herramientas y recursos disponibles.

En segundo lugar: Desarrollo del algoritmo de generación de melodías. Utilizaremos técnicas de aprendizaje automático y modelado generativo para crear un sistema capaz de componer melodías originales y variadas.

En tercer lugar: Implementación de la armonización de las melodías generadas. En esta etapa, diseñaremos y desarrollaremos heurísticas y reglas musicales para armonizar las melodías de manera coherente y agradable al oído.

En cuarto lugar: Selección y configuración de instrumentos. Investigaremos y seleccionaremos una variedad de instrumentos musicales virtuales que se ajusten a las temáticas predefinidas de los videojuegos. Configuraremos estos instrumentos para que reproduzcan las melodías generadas de manera convincente al oído humano.

En último lugar: Desarrollo de la aplicación de escritorio. Utilizaremos las tecnologías adecuadas para desarrollar una interfaz de usuario intuitiva y amigable que permita a los usuarios generar, armonizar y sonorizar sus propias composiciones musicales para videojuegos.


% Al comentar la siguiente línea, el pdf no compila
\lipsum[0]